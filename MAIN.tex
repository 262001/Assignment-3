\documentclass[journal,12pt,two column]{IEEEtran}
\usepackage[cmex10]{amsmath}
\usepackage[utf8]{inputenc}
\usepackage{graphicx}
\usepackage{caption}
\usepackage{subcaption}
\usepackage{stfloats}
\usepackage{amsmath}
\usepackage{amssymb}
\usepackage{amsfonts}
\usepackage{nopageno}
\usepackage[margin=1in]{geometry}
\usepackage{graphicx}
\usepackage{float}
\usepackage{multicol}
\usepackage{hyperref}
\setlength{\parindent}{0em}
\usepackage{color}
\usepackage{comment}

\usepackage[T1]{fontenc}
\setlength{\parindent}{0pt}


\title{AI1103 - Assignment 3}
\author{Monika Kharadi - CS20BTECH11026}
\date{March 2021}
\begin{document}
\maketitle
\section*{\large\textbf{Problem :}}
3. The mean and variance, respectively of a binomial distribution for n independent trials with the probability of success as p, are \\
(A) $\sqrt{np}$, np(1-2p)\\
(B) $\sqrt{np}$, $\sqrt{np(1-p)}$\\
(C) np, np \\
(D) np, np(1-p)
\section*{\large\textbf{Solution}}
Let $X_1,X_2,X_3,....,X_n$ be the random variable for n independent trials \\ \\
Expected Value for n trials : 
\begin{align}
E(X_i) &= X_i\cdot p_i \nonumber\\
E(X_i) &= p
\end{align}
We know that,
\begin{align}
E(X) &= \sum_{i=1}^n E (X_i) \nonumber\\
E(X) &= np
 \end{align}
Mean of a binomial distribution for n independent trials is \textbf{np}.\\ \\
Now,
\begin{align}
E(X_i^2) &= X_i^2\cdot p_i \nonumber\\
E(X_i^2) &= p
\end{align}
For variance,
\begin{align}
    Var(X_i) &= E(X_i^2) - E(X_i)^2 \\
    Var(X_i) &= p - p^2 
    \end{align}
We can add Var($X_i$)  to get Var(X)  as these are independent trials
\begin{align}
Var(X) &= \sum_{i=1}^n Var(X_i) \nonumber \\
Var(X) &= n(p - p^2)\nonumber\\
Var(X) &= np(1-p)
\end{align}
Variance of a binomial distribution for n independent trials is \textbf{np(1-p)}.\\ \\
Hence, (D) is correct option.


\end{document}



